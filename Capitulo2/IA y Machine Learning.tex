
%%%%%%%%%%%%%%%%%%%%%%%%%%%%%%%%%%%%%%%%%%%%%%%%%%%%%%%%%%%%%%%%%%%%%%%%%
%           Capítulo 2: MARCO TEÓRICO - REVISIÓN DE LITERATURA
%%%%%%%%%%%%%%%%%%%%%%%%%%%%%%%%%%%%%%%%%%%%%%%%%%%%%%%%%%%%%%%%%%%%%%%%%

\chapter{IA y Machine Learning}
En este capítulo, normalmete se ponen todas las ecuaciones que se van a usar en la tesis, así ya nomás se hace rferencia a la ecuación tal o "como se vió en el capítulo 2", y esas cosas [\citep{Pau86}].

\section{Definición}
\label{sec:SistemasL}
\blindtext
\section{Diseño}
Como se dijo en la sección \ref{sec:SistemasL} los sistemas L [\cite{Peitgen2004}] son...CAMBOOOPOO
\blindtext

\section{Analisis Exploratorio}
De acuerdo con [\citep{KaplanGlass1995}] un autómata celular es ...

%inserción de codigo de Matlab
%Es conveniente sangrarlo (los de proteco dicen "indentarlo") para que no se encime con los números  de las líneas a la izquierda
\begin{lstlisting}[frame=single]
    % Declaracion de las variables simbolicas
    syms u z1 z2 z3 z4 J m M g l 
    % Matrices involucradas
    E = [J+m*l*l m*l*cos(z1);m*l*cos(z1) M+m] 
    F = [m*g*l*sin(z1);u+m*l*(z3*z3)*sin(z1)] 
    % Despeje
    V = E\F
\end{lstlisting}

\section{Preparación de los datos}

\blindtext

\subsection{Cumplimiento de supuesto}

\blindtext

\subsection{Muestreo y submuestreo}

\blindtext

\subsection{PCA}

\blindtext

\subsection{Reducción de dimensiones}

\blindtext

\section{Aprendizaje Supervisado}

\blindtext

\subsection{Clustering}

\blindtext

\subsubsection{Kmeans}

\blindtext

\subsubsection{Clustering GMM}

\blindtext

\section{Aprendizaje No Supervisado}

\blindtext

\subsection{Clasificación Binaria}

\blindtext

\subsubsection{Regresión Logistica}

\blindtext

\subsubsection{K Nearest Neighbors}

\blindtext

\subsubsection{Random Forest}

\blindtext

\section{Implementación de los modelos}

\blindtext

\subsection{Ajuste de los modelos}

\blindtext

\subsection{Metricas de Ajuste (Evaluación)}

\blindtext